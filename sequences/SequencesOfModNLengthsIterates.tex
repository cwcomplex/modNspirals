\documentclass[11pt]{amsart}
\usepackage{geometry}                % See geometry.pdf to learn the layout options. There are lots.
\geometry{letterpaper}                   % ... or a4paper or a5paper or ... 
%\geometry{landscape}                % Activate for for rotated page geometry
%\usepackage[parfill]{parskip}    % Activate to begin paragraphs with an empty line rather than an indent
\usepackage{graphicx}
\usepackage{amssymb}
\usepackage{epstopdf}
\usepackage{graphicx}
\usepackage{float}
\usepackage{hyperref}
\DeclareGraphicsRule{.tif}{png}{.png}{`convert #1 `dirname #1`/`basename #1 .tif`.png}

\title{Sequences from (mod N) Spiral Lengths and Iterates}
\author{Andrew R. Reiter\\arr@watson.org}
%\date{}                                           % Activate to display a given date or no date

\begin{document}
\maketitle

\section{Introduction}
In reading the brief note from the author \cite{ARRmodN}, the reader will have learned about the construction of (mod N) spirals defined on a square lattice. In this note, we focus on the sequences generated from the length and iteration formulas for complete spiral $Ond^k_N$.

\section{Sequences from Length Formula}
Let us first recall the formula for the length of the sides of the complete spiral $Ond^k_N$ to be $\lambda = \frac{kN}{\sqrt{s}}$ where $s$ is the \textit{greatest square divisor} of $N$. Let $s = g(N)$ where $g : \mathbb{N} \mapsto \{ x \in \mathbb{N} : \forall y \in \mathbb{N}\ s.t.\ y^2 = x \}$ to represent the function that returns the greatest square divisor of $N$.

We can look at $\lambda$ and see two ways in which to start generating sequences. The first is to fix $N$, and thus $g(N)$, and iterate through values $k=1, 2, 3, \ldots$. In fixing $N$, the formula results in $\lambda = kC$ where $C = \frac{N}{\sqrt(g(N))}$ is constant. Thus, $\lambda$ is a line, and the set of $\lambda$'s for all $N$ is a set of lines. Some of the lines will be relatively parallel. 

\begin{figure}[H]
\centering
\includegraphics[scale=0.7]{len_N.png}
\caption{Lengths for $Ond^k_N$ with fixed $N$}
\label{fig:len_N}
\end{figure}

The second method is to fix $k$ and iterate through values $N=2, 3, 4, \ldots$, thus producing a set of values $\{ g(2), g(3), g(4), \ldots \}$. In fixing $k$, the formula is still $\lambda = \frac{kN}{\sqrt{g(N)}}$. The method $\frac{N}{\sqrt{g(N)}}$ piece of this equation results in the non-linearity of $lambda$ for this case. 

\begin{figure}[H]
\centering
\includegraphics[scale=0.7]{len_k.png}
\caption{Lengths for $Ond^k_N$ with fixed $k$}
\label{fig:len_k}
\end{figure}

%\subsection{}

\section{Sequences from Iteration Forumla}
Let us recall the formula for the iteration count of the complete spiral $Ond^k_N$ to be $\xi = \frac{k^2N}{g(N)}$, where again $g$ is the greatest square divisor map. In a similar manner to the above, we can generate sequences in two ways: by fixing $N$ and iterating through $k \ge 1$ or by fixing $k$ and iterating through values of $N \ge 2$. If we look at the former and fix $N$ then we are also fixing $g(N)$ and leads to $\xi = Ck^2$

\begin{figure}[H]
\centering
\includegraphics[scale=0.7]{iter_N.png}
\caption{Iteration counts for $Ond^k_N$ with fixed $N$}
\label{fig:iter_N}
\end{figure}

The second method is to fix $k$, which results in $\xi = \frac{CN}{g(N)}$ where $C = k^2$. Similar to above, the portion of $\xi$ that is making this non-linear is $\frac{N}{g(N)}$.

\begin{figure}[H]
\centering
\includegraphics[scale=0.7]{iter_k.png}
\caption{Iteration counds for $Ond^k_N$ with fixed $k$}
\label{fig:iter_k}
\end{figure}

\begin{thebibliography}{1}
\bibitem{ARRmodN} A. Reiter, ``On (mod N) Spirals'', \url{http://www.cw-complex.com/modNspirals/modNspirals.pdf}
\end{thebibliography}
\end{document}  