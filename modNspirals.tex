
%%%%  RY edits:
%%
%%  I edited the main section(s): haven't looked over the introduction
%%  yet.  Essentially dealt with issues 1), 2) and 3) in
%%  RY_comments.txt.  I saw that you had already included my
%%  suggestions in 4) in the "future work" section.  --  It may be
%%  worth writing down the answer for a three-D lattice now!
%%  Regarding the length and diagrams:  it is pretty much exactly 4
%%  pages as is:  I'll play with tikz for a bit and send it on if I
%%  make progress ...  



\documentclass[11pt,reqno]{amsart}
\usepackage{geometry}                % See geometry.pdf to learn the layout options. There are lots.
\geometry{letterpaper}                   % ... or a4paper or a5paper or ... 
%\geometry{landscape}                % Activate for for rotated page geometry
%\usepackage[parfill]{parskip}    % Activate to begin paragraphs with an empty line rather than an indent
\usepackage{graphicx}
\usepackage{amssymb}
\usepackage{epstopdf}
\usepackage{hyperref}
\usepackage{graphicx}
\usepackage{subfig}
\usepackage{tikz}
\usepackage{enumerate}


\theoremstyle{mydef}
\newtheorem{definition}{Definition}
\newtheorem{conj}{Conjecture}[section]
\newtheorem{thm}{Theorem}[section]
\def\ZZ{\mathbb{Z}}


\DeclareGraphicsRule{.tif}{png}{.png}{`convert #1 `dirname #1`/`basename #1 .tif`.png}

\title{On (mod n) Spirals}
\author{Andrew Reiter}
\author{Robin Young}
\email{areiter@veracode.com, young@math.umass.edu}
%\date{}                                           % Activate to display a given date or no date

%%%%%%  RY: this line gets removed!
%\def\includegraphics[#1]{\fbox}

\begin{document}
\maketitle
\section{Introduction}

This note is intended to introduce the process of constructing (mod n) spirals and the idea of a \textit{complete spiral}.  It also introduces a few theorems, with proof, related to patterns seen in the construction of \textit{complete spirals}; regarding the lengths of sides, iteration counts, and ending corners of these objects. Further, from the theorem on iteration counts one sees that these complete spirals provide a (manual) process for discovering the \textit{greatest square divisor} of integers $n \ge 2$. Lastly, grayscale visualizations of a number \textit{complete spirals} were generated to further the investigatory process and are shared herein. While not inspired by Ulam's Spiral \cite{Ulam}, the construction of  a (mod n) spiral is similar in nature. The author's are unaware of other work or literature on this topic, so have no references other the couple related to code; in part this is why the note has been written. Further, the investigation provided some bit 
 of fun which has made it worthwhile.

\section{Spiral Construction}

We describe the construction of a (mod $n$) spiral and introduce the
notation we will use to analyze these.  For a fixed integer $n \ge 2$,
we will be working with the additive group
$\ZZ_n=\mathbb{Z}/n\mathbb{Z} = \{ 0, 1, ..., n-1 \}$.  Let $L$
be a square lattice, which we can take to be $\ZZ^2$.  Denoting
the origin by $l_1 = (0,0)$, we build the spiral by enumerating the
lattice sites and assigning numbers from $\mathbb{Z}_n$ in turn.  We
spiral in a clockwise direction starting in the direction of the
$x$-axis, so that $l_2 = (1,0)$, $l_3 = (1,-1)$, and the next four
lattice points are $\{(0, -1),(-1, -1),(-1, 0),(-1, 1)\}$,
respectively.  In general, once $l_j$ has been assigned, and having
chosen a `direction' by moving from $l_{j-1}$, the next site $l_{j+1}$
is the site to the right if this is not yet accounted for, or the site
ahead otherwise.

Having enumerated the lattice as above, we now assign values from
$\ZZ_n$ as follows: to the origin we assign $l_1^* = 0$, and we then
count in $\ZZ_n$: $l_2^* = 1$, $\l_3^* = 2$, etc., and cycling back to
$0$ after $n$ steps, so that $l_{n+1}^* = 0$.  Continuing in this way,
we `count' all lattice sites (mod $n$).  In general, this yields
\[
  l_j^* = j-1 \!\!\!\pmod n \quad\text{at site}\quad l_j.
\]


\begin{figure}[h]
\[  \begin{array}{c}
\boxed{0}
\end{array} 
\rightarrow
%
\begin{array}{cc}
\boxed{0} & 1 \\
\  & \ 
\end{array}
\rightarrow
\begin{array}{cc}
\boxed{0} & 1 \\
\  & 2
\end{array}
\rightarrow
\begin{array}{cc}
\boxed{0} & 1 \\
0 & 2
\end{array}
\rightarrow
\begin{array}{ccc}
\  & \boxed{0} & 1 \\
1 & 0 & 2
\end{array}
\rightarrow
\begin{array}{ccc}
2 & \boxed{0} & 1 \\
1 & 0 & 2
\end{array}
\rightarrow
\begin{array}{ccccc}
\  & 0 & \cdot  & \cdot & \cdot \\
\  & 2 & \boxed{0} & 1 & \cdot \\
\  & 1 & 0 & 2 & \cdot \\
\cdot & \cdot & \cdot & \cdot & \cdot
\end{array}
\]
\caption{Building a (mod 3) spiral}
\label{fig:mod3spiral}
\end{figure}

%Choose a point $l_0 \in L$ and it will be the starting point of the spiral; choose $l_0 = (0, 0)$, the origin. Let $l^*_0$ be the value assigned to the point $l_0$. We want to assign the first element of $\mathbb{Z}/n\mathbb{Z}$ to the first point in the spiral, thus, we assign $l^*_0 = 0$. Now, we wish to choose a second point in the spiral, $l_1$, by moving in the positive $x$ direction by one unit and assign it the second value of $\mathbb{Z}/n\mathbb{Z}$. So we have $l_1 = (1, 0)$ and $l^*_1 = 1$.  We will continue to choose points $l_i$ and assigning values from $\mathbb{Z}/n\mathbb{Z}$ in an increasing order.



%To continue this process, let us try to visualize us being on the lattice work itself. Orient your mind so that previous move to $l_1$ was you ``moving forward''. With current position $l_1$, check the lattice point to your right: if it is assigned a value already, move ``forward'' to the next lattice point. However, if the lattice point to the right was \emph{not} occupied, then you ``turn right'' and ``move forward'' to occupy it. You are now at $l_2$ and have $l^*_2 = 2$ and, based on above, $l_2 = (1, -1)$. . You repeat this pattern of ``looking right'' to determine if you should ``turn right'' or keep ``moving forward'' and then assigning values from $\mathbb{Z}/n\mathbb{Z}$. Continuing on with the lattice points we find $\{ l_i \}^{6}_{i=3} = \{  (0, -1), (-1, -1), (-1, 0), (-1, 1) \}$


%At some point, the process will reach a lattice point $l_i$ in which $l^*_{i-1} = n-1$. This is means an \textit{iteration of (mod n)} has occurred and we assign $l^*_i = 0$. This restarts going through values of $\mathbb{Z}/n\mathbb{Z}$. The (mod n) spiral is the set of tuples $\{ (l_0, l^*_0), (l_1, l^*_1), ... \}$ 

%The purpose of inventing notation for the lattice points is to help give a scalar mapping into the lattice based on the spiral. Currently it is not used, but could be used in research investigating patterns on the diagonals of generated spirals. In looking at the construction of a spiral based on $\mathbb{Z}/10\mathbb{Z}$ in figure \ref{fig:mod10}, with the starting point $l_0$ with value $l^*_0 = 0$ marked by a box, one can begin to see the diagonal patterns one might wish to investigate.


%\section{complete spiral}
The first author's initial interest in generating spirals of
increasing size for various $\ZZ_n$ was to investigate some of the
% diagonal patterns as referred to above and 
patterns seen in figure \ref{fig:mod10}.  In an attempt to describe
the patterns, he introduced the following definition:
%In thinking about how to do this in strict manner, the following
%definitions were created after some thought:

\begin{figure}[h]
\centering
\begin{tikzpicture}

%% make nodes xiyj
\foreach \nx in {-6,...,6}
{
  \foreach \ny in {-6,...,6} 
  {
      \node (x\nx y\ny) at (\nx,\ny) {};
  }
}

\node at (0,0) {0};
\node at (1,0) {1};
\node at (1,-1) {2};
\node at (0,-1) {3};
\node at (-1,-1) {4};
\node at (-1,0) {5};
\node at (-1,1) {6};
\node at (0,1) {7};
\node at (1,1) {8};
\node at (2,1) {9};
\node at (2,0) {0};
\node at (2,-1) {1};
\node at (2,-2) {2};
\node at (1,-2) {3};
\node at (0,-2) {4};
\node at (-1,-2) {5};
\node at (-2,-2) {6};
\node at (-2,-1) {7};
\node at (-2,0) {8};
\node at (-2,1) {9};
\node at (-2,2) {0};
\node at (-1,2) {1};
\node at (0,2) {2};
\node at (1,2) {3};
\node at (2, 2) {4};
\node at (3, 2) {5};
\node at (3, 1) {6};
\node at (3, 0) {7};
\node at (3, -1) {8};
\node at (3, -2) {9};
\node at (3, -3) {0};
\node at (2, -3) {1};
\node at (1, -3) {2};
\node at (0, -3) {3};
\node at (-1, -3) {4};
\node at (-2, -3) {5};
\node at (-3, -3) {6};
\node at (-3, -2) {7};
\node at (-3, -1) {8};
\node at (-3, 0) {9};
\node at (-3, 1) {0};
\node at (-3, 2) {1};
\node at (-3, 3) {2};
\node at (-2, 3) {3};
\node at (-1, 3) {4};
\node at (0, 3) {5};
\node at (1, 3) {6};
\node at (2, 3) {7};
\node at (3, 3) {8};
\node at (4, 3) {9};
\node at (4, 2) {0};
\node at (4, 1) {1};
\node at (4, 0) {2};
\node at (4, -1) {3};
\node at (4, -2) {4};
\node at (4, -3) {5};
\node at (4, -4) {6};
\node at (3, -4) {7};
\node at (2, -4){8};
\node at (1, -4) {9};
\node at (0, -4) {0};
\node at (-1, -4) {1};
\node at (-2, -4) {2};
\node at (-3, -4) {3};
\node at (-4, -4) {4};
\node at (-4, -3) {5};
\node at (-4, -2) {6};
\node at (-4, -1) {7};
\node at (-4, 0) {8};
\node at (-4, 1) {9};
\node at (-4, 2) {0};
\node at (-4, 3) {1};
\node at (-4, 4) {2};
\node at (-3, 4) {3};
\node at (-2, 4) {4};
\node at (-1, 4) {5};
\node at (0, 4) {6};
\node at (1, 4) {7};
\node at (2, 4) {8};
\node at (3, 4) {9};
\node at (4, 4) {0};
\node at (5, 4) {1};
\node at (5, 3) {2};
\node at (5, 2) {3};
\node at (5, 1) {4};
\node at (5, 0) {5};
\node at (5, -1) {6};
\node at (5, -2) {7};
\node at (5, -3) {8};
\node at (5, -4) {9};
\node at (5, -5) {0};
\node at (4, -5) {1};
\node at (3, -5) {2};
\node at (2, -5) {3};
\node at (1, -5) {4};
\node at (0, -5) {5};
\node at (-1, -5) {6};
\node at (-2, -5) {7};
\node at (-3, -5) {8};
\node at (-4, -5) {9};
\node at (-5, -5) {0};
\node at (-5, -4) {1};
\node at (-5, -3) {2};
\node at (-5, -2) {3};
\node at (-5, -1) {4};
\node at (-5, 0) {5};
\node at (-5, 1) {6};
\node at (-5, 2) {7};
\node at (-5, 3) {8};
\node at (-5, 4) {9};
\node at (-5, 5) {0};
\node at (-4, 5) {1};
\node at (-3, 5) {2};
\node at (-2, 5) {3};
\node at (-1, 5) {4};
\node at (0, 5) {5};
\node at (1, 5) {6};
\node at (2, 5) {7};
\node at (3, 5) {8};
\node at (4, 5) {9};
\node at (5, 5) {0};
\node at (6, 5) {1};
\node at (6, 4) {2};
\node at (6, 3) {3};
\node at (6, 2) {4};
\node at (6, 1) {5};
\node at (6, 0) {6};
\node at (6, -1) {7};
\node at (6, -2) {8};
\node at (6, -3) {9};
\node at (6, -4) {0};
\node at (6, -5) {1};
\node at (6, -6) {2};
\node at (5, -6) {3};
\node at (4, -6) {4};
\node at (3, -6) {5};
\node at (2, -6) {6};
\node at (1, -6) {7};
\node at (0, -6) {8};
\node at (-1, -6) {9};
\node at (-2, -6) {0};
\node at (-3, -6) {1};
\node at (-4, -6) {2};
\node at (-5, -6) {3};
\node at (-6, -6) {4};
\node at (-6, -5) {5};
\node at (-6, -4) {6};
\node at (-6, -3) {7};
\node at (-6, -2) {8};
\node at (-6, -1) {9};
\node at (-6, 0) {0};
\node at (-6, 1) {1};
\node at (-6, 2) {2};
\node at (-6, 3) {3};
\node at (-6, 4) {4};
\node at (-6, 5) {5};
\node at (-6, 6) {6};
\node at (-5, 6) {7};
\node at (-4, 6) {8};
\node at (-3, 6) {9};
\node at (-2, 6) {0};
\node at (-1, 6) {1};
\node at (0, 6) {2};
\node at (1, 6) {3};
\node at (2, 6) {4};
\node at (3, 6) {5};
\node at (4, 6) {6};
\node at (5, 6) {7};
\node at (6, 6) {8};



\draw[yellow] (-.5,-.5) -- (0.5,-.5) -- (0.5,.5);
\draw[green] (-.5,-1.5) -- (-.5,.5) -- (1.5,.5);
\draw[blue] (-1.5,-1.5) -- (1.5,-1.5) -- (1.5,1.5);
\draw[magenta] (-1.5,-2.5) -- (-1.5,1.5) -- (2.5,1.5);
\draw[red] (-2.5,-2.5) -- (2.5,-2.5) -- (2.5,1.5);


\end{tikzpicture} 
\caption{a (mod 10) spiral with starting 0 enblocked}
\label{fig:mod10}
\end{figure}


\begin{definition}%[complete spiral]
  A \emph{complete spiral} occurs when in the spiral construction, the
  partially completed spiral forms a square, and the last (mod $n$)
  value assigned is $l^*_i = n-1$, so all of $\ZZ_n$ has been used an
  integer number of times.
\end{definition}

We use the following notation: denote the first complete spiral
achieved by $Ond^1_n$; subsequent complete spirals are denoted
$Ond^k_n$, $k = 2,3,\dots$.  The number of times $\ZZ_n$ is used in
order to complete the $k^{th}$ square is the iteration count.  The
word ``ond'' is ``spiral'' in Swahili.

\begin{figure}[h]
\centering
\subfloat[][]
{$\begin{array}{ccc}
0 & 1 & \dot{2} \\
2 & \boxed{0} & 1 \\
1 & 0 & 2
\end{array}$\label{ond13}}
\qquad
\subfloat[][]
{$\begin{array}{cccccc}
2 & 0 & 1 & 2 & 0 & 1 \\
1 & 0 & 1 & 2 & 0 & 2 \\
0 & 2 & \boxed{0} & 1 & 1 & 0 \\
2 & 1 & 0 & 2 & 2 & 1 \\
1 & 0 & 2 & 1 & 0 & 2 \\
\dot{2} & 1 & 0 & 2 & 1 & 0
\end{array}$\label{ond23}}
\qquad
\subfloat[][]
{$\begin{array}{cc}
\boxed{0} & 1 \\
\dot{3} & 2
\end{array}$\label{ond14}}
\qquad
\subfloat[][]
{$\begin{array}{cccc}
2 & 3 & 0 & 1\\
1 & \boxed{0} & 1 & 2\\
0 & 3 & 2 & 3\\
\dot{3} & 2 & 1 & 0
\end{array}$\label{ond24}}
\caption{ \protect\subref{ond13} $Ond^1_3$,  \protect\subref{ond23} $Ond^2_3$, \protect\subref{ond14} $Ond^1_4$,  \protect\subref{ond24} $Ond^2_4$}
\label{fig:mod34}
\end{figure}

Figure \ref{fig:mod34} shows $Ond_n^k$ for $k=1,2$ and $n=1,2$.  We
see that $Ond^1_3$ is a $3\times3$ square and has iteration count 3,
and $Ond^2_3$ is a $6\times6$ square with 12 iterations.  Similarly,
$Ond^1_4$ is $2\times2$ with a single iteration, and $Ond^2_4$ is
$4\times4$ with 4 iterations. The first element, 0, of each spiral in the
figures is in a box and last element of the spiral marked with a dot.

The author generated several spirals with pencil and paper in a
methodical manner to create sets of complete spirals for various
values of $n$ and $k$.  While tedious and not exactly related to the
initial goal of looking at diagonal patterns, it helped to realize
there seemed to be patterns found in the construction of the spirals.
Specifically in the sizes of the complete spirals, iteration counts,
and where the last lattice point rested; this led the author to
investigate what the patterns were.

\subsection{On the side lengths and iterate counts of $Ond^k_n$}

In order to investigate these patterns, the author determined that
more data was needed and, due to the tedium of pencil and paper, a
program should be written to generate complete spirals
\cite{PySquare}.  This allowed the author to generate a larger number
of complete spirals and collect data on lengths, iterations, and
ending points.

In looking at the initial data, for small choices of $n, k$, a few
patterns in tuples of (lengths, iterations) were found, including
$(kn, k^2 n)$, $(\frac{kn}{2}, \frac{k^2 n}{4})$, and $(\sqrt{n}k,
k^2)$.  To establish the pattern, the prime factorization of $n$ for
each complete spiral $Ond^k_n$ with the lengths and iteration data was
generated for analysis.  From this data, the author determined there
was a relation involved with finding the greatest square divisor of
$n$, which led to the following observations.

\begin{thm}%[Length of Sides of $Ond^k_n$]
\label{lenthm}
Let $s$ denote the greatest square divisor of $n$.
The complete spiral $Ond^k_n$ has the following structure:
\begin{enumerate}[(i)]
\item If $\lambda$ is the length of the sides of $Ond^k_n$, then
\begin{equation}
  \lambda = \frac{kn}{\sqrt{s}}.
\label{lambda}
\end{equation}
\item If $\xi$ is the iteration count of $Ond^k_n$, then
\begin{equation}
  \xi = \frac{k^2n}{s}.  
\label{xi}
\end{equation}
\item If $l_{max} \in L$ is the last lattice point in the complete
  spiral $Ond^k_n$, then $l_{max}$ is either the top-right corner or
  bottom-left corner of the square.  If both $n$ and $k$ are odd, then
  $l_{max}$ will be the top-right corner of $Ond^k_n$.  In all other
  cases, $l_{max}$ will be the bottom-left corner.
\end{enumerate}
\end{thm}

One will note that in the case where $n$ is square-free, then $s=1$
and \eqref{lambda} and \eqref{xi} reduce to $\lambda = k n$ and $\xi =
k^2 n$, respectively. These can be considered an upper bound for all
cases.  One might also consider these to be the least robust in the
case of length and iteration counts.

A most interesting aspect to this process of complete spiral
construction is the connection to the greatest square divisor of some
integer $n$.  To see this more clearly, choose some $n$ and construct,
by hand, the spiral $Ond^1_n$.  At this point, you know $\lambda$ and
$\xi$, so pick one, substitute $k = 1$ and solve for $s$.  So from a
constructivist approach we find the greatest square divisor.

%\subsection{Grayscale Visualization}

\begin{figure}[h]
\centering
\includegraphics[scale=.4]{N31k39.png}
\caption{grayscale visualization of $Ond^{39}_{31}$}
\label{fig:viz3139}
\end{figure}

In the process of investigating sizes and iteration counts of
$Ond^k_n$, the first author implemented a method to map the generated
spirals to grayscale images. This works best for positive integers,
$n$, less than 256.  The map $f : \ZZ_n \to \ZZ_{256}$ is defined by a
scaled floor function,
\[
   f(j) = \alpha j, \quad \text{where} \quad
   \alpha = \left\lfloor \frac{255}n \right\rfloor
\]
The function $f$ thus maps $\ZZ_n$ to a set of brightness values which
are used to generate grayscale images.



An example of this is $Ond^{39}_{31}$ seen in figure \ref{fig:viz3139}. One can
see a complex pattern emerging.  Further, if one zooms
in and out on the image, various different patterns emerge that seem
to play tricks on the eye.  The image may also be view at
\cite{GraySquare} and is the file \textit{N=31\_k=39-grey.png} which
makes it easier for testing the zoom in/out.  Further, at this
location, one may find the generated images for $Ond^k_n$ for $n=2, 3,
\ldots, 31$ and $k=1,2, \ldots, 50$.

\

\begin{proof}[Proof of the Theorem]
  Recall that $n\ge 2$ is fixed.  Let $\lambda$ be the length of the
  side of a complete spiral, and let $\xi$ be the corresponding
  iteration count.  Since the spiral is square, we must have $\xi n =
  \lambda^2$, and so $n \vert \lambda^2$.  Now any prime which divides
  $n$ must also divide $\lambda^2$, and must thus divide $\lambda$.

  Since $s$ is the greatest square divisor of $n$, we can uniquely
  write $n = q^2_1 q_2$ where $q_2$ is square-free, with $s = q_1^2$.
  It follows that $q_1^2|\lambda^2$, so $q_1|\lambda$ and $q_2 |
  (\lambda/q_1)^2$.  Since $q_2$ is square-free, $q_2|(\lambda/q_1)$
  and we have $\lambda = k q_1 q_2$ for some integer $k$.  Clearly
  $n|\lambda^2$ for any such $\lambda$, and we conclude that $\lambda
  = k q_1 q_2$ is the side of the $k$-th complete spiral $Ond^k_n$.

  Equation \eqref{lambda} now follows since $\sqrt{s} = q_1$, and
  \eqref{xi} follows since $\xi = \lambda^2/n$.  Finally, $(iii)$
  follows since $l_{max} = \lambda^2$ is even unless $k$, $q_1$ and
  $q_2$ are all odd, and by construction $l_{max}$ represents the
  top-right corner of the square if it is odd, and the bottom-left
  corner if it is even.
\end{proof}


\section{Generalizations}

It would be of interest to think about looking at how spiraling might
work for cubes or higher dimensional squares, since the lattice is the
basis for the spiral. Further, it might be interesting to see how one
might spiral with other objects that will tile the plane,
i.e. triangles and hexagons.

Lastly, characterizing the visualizations and interpreting their
relations to the process would be a neat endeavor.

The diagonal patterns in the larger $k$ $Ond^k_n$ spirals would make
for interesting to see how to categorize and determine their
generators. In a private conversation \cite{Kusner} suggested also
looking at the angular velocity of spiral growth, which might be fun.

\vspace{12pt}\noindent\textbf{Acknowledgments:}\quad
The first author recognizes Veracode Inc for providing time to think
about these questions and Jared Carlson of Veracode for some initial
discussion.  He would also like to thank Christoph Pacher for feedback
and thinking about other tilings of the plane and Geoff Bailey of the
University of Sydney for comments on what were conjectures.

\begin{thebibliography}{1}

\bibitem{Ulam} Wolfram Mathworld, ``Prime Spirals'',
  \url{http://mathworld.wolfram.com/PrimeSpiral.html}
\bibitem{PySquare} A. Reiter,
  \url{https://github.com/cwcomplex/modNspirals}
\bibitem{GraySquare} A. Reiter,
  \url{https://github.com/cwcomplex/modNspirals/tree/master/somegrey}
\bibitem{sqmaxes} A. Reiter,
  \url{https://github.com/cwcomplex/modNspirals/blob/master/squareoff.py}
\bibitem{Kusner} R. Kusner, A. Reiter, \textit{private email
    correspondence}

\end{thebibliography}


\end{document}  
