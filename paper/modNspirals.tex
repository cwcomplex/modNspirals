\documentclass[11pt,reqno]{amsart}
\usepackage{geometry}                % See geometry.pdf to learn the layout options. There are lots.
\geometry{letterpaper}                   % ... or a4paper or a5paper or ... 
%\geometry{landscape}                % Activate for for rotated page geometry
%\usepackage[parfill]{parskip}    % Activate to begin paragraphs with an empty line rather than an indent
\usepackage{graphicx}
\usepackage{amssymb}
\usepackage{epstopdf}
\usepackage{hyperref}
\usepackage{graphicx}
\usepackage{subfig}
\usepackage{tikz}
\usepackage{enumerate}


\theoremstyle{mydef}
\newtheorem{definition}{Definition}
\newtheorem{conj}{Conjecture}[section]
\newtheorem{thm}{Theorem}[section]
\def\ZZ{\mathbb{Z}}


\DeclareGraphicsRule{.tif}{png}{.png}{`convert #1 `dirname #1`/`basename #1 .tif`.png}

\title{On (mod $n$) Spirals}
\author{Andrew Reiter}
\author{Robin Young}
\email{areiter@veracode.com, young@math.umass.edu}

%%%%%%  RY: this line gets removed!
%\def\includegraphics[#1]{\fbox}

\begin{document}
\maketitle
\section{Introduction}

This note is intended to introduce the process of constructing (mod $n$) spirals and the idea of a complete spiral.  The construction of a (mod $n$) spiral is similar in nature to Ulam's Spiral \cite{Ulam}. The note also introduces a theorem related to patterns seen regarding the lengths of sides, iteration counts, and ending corners of these objects, and provides a proof. This process also provides a deterministic process for discovering the greatest square divisor of integers $n \ge 2$. Grayscale visualizations of these spirals have been generated in order to  further illuminate understanding and interest in these objects. Generalizations to higher dimensions and spirals of other shapes are given. The author's are unaware of other work or literature on this topic, so have no references other than the couple related to code; in part this is why the note has been written and offered to share as it is a candidate for use in classroom learning.



\section{Spiral Construction}

We describe the construction of a (mod $n$) spiral and introduce the
notation we will use to analyze these.  For a fixed integer $n \ge 2$,
we will be working with the additive group
$\ZZ_n=\mathbb{Z}/n\mathbb{Z} = \{ 0, 1, ..., n-1 \}$.  Let $L$
be a square lattice, which we can take to be $\ZZ^2$.  Denoting
the origin by $l_1 = (0,0)$, we build the spiral by enumerating the
lattice sites and assigning numbers from $\mathbb{Z}_n$ in turn.  We
spiral in a clockwise direction starting in the direction of the
positive $x$-axis, so that $l_2 = (1,0)$, $l_3 = (1,-1)$, and the next four
lattice points are $\{(0, -1),(-1, -1),(-1, 0),(-1, 1)\}$,
respectively.  In general, once $l_j$ has been assigned, and having
chosen a `direction' by moving from $l_{j-1}$, the next site $l_{j+1}$
is the site to the right if this is not yet accounted for, or the site
ahead otherwise.

\begin{figure}[h]
\[  \begin{array}{c}
\boxed{0}
\end{array} 
\rightarrow
%
\begin{array}{cc}
\boxed{0} & 1 \\
\  & \ 
\end{array}
\rightarrow
\begin{array}{cc}
\boxed{0} & 1 \\
\  & 2
\end{array}
\rightarrow
\begin{array}{cc}
\boxed{0} & 1 \\
0 & 2
\end{array}
\rightarrow
\begin{array}{ccc}
\  & \boxed{0} & 1 \\
1 & 0 & 2
\end{array}
\rightarrow
\begin{array}{ccc}
2 & \boxed{0} & 1 \\
1 & 0 & 2
\end{array}
\rightarrow
\begin{array}{ccccc}
\  & 0 & \cdot  & \cdot & \cdot \\
\  & 2 & \boxed{0} & 1 & \cdot \\
\  & 1 & 0 & 2 & \cdot \\
\cdot & \cdot & \cdot & \cdot & \cdot
\end{array}
\]
\caption{Building a (mod 3) spiral}
\label{fig:mod3spiral}
\end{figure}

Having enumerated the lattice as above, we now assign values from
$\ZZ_n$ as follows: to the origin we assign $l_1^* = 0$, and we then
count in $\ZZ_n$: $l_2^* = 1$, $\l_3^* = 2$, etc., and cycling back to
$0$ after $n$ steps, so that $l_{n+1}^* = 0$.  Continuing in this way,
we `count' all lattice sites (mod $n$).  In general, this yields
\[
  l_j^* = j-1 \!\!\!\pmod n \quad\text{at site}\quad l_j.
\]




The first author's initial interest in generating spirals of
increasing size for various $\ZZ_n$ was to investigate some of the
patterns seen when using $\ZZ_{10}$ as the set of values.  In an attempt to describe
the patterns, he introduced the following definition:



\begin{definition}%[complete spiral]
  A \emph{complete spiral} occurs when in the spiral construction, the
  partially completed spiral forms a square, and the last (mod $n$)
  value assigned is $l^*_i = n-1$, so all of $\ZZ_n$ has been used an
  integer number of times.
\end{definition}

We use the following notation: denote the first complete spiral
achieved by $Ond^1_n$; subsequent complete spirals are denoted
$Ond^k_n$, $k = 2,3,\dots$.  The number of times $\ZZ_n$ is used in
order to complete the $k^{th}$ square is the iteration count.  The
word ``ond'' is ``spiral'' in Swahili.

REVIEW FIGURE REMOVAL AND PARAGRAPH REMOVAL
%%% THIS RESULTED FROM FIGURE REMOVAL
%Figure \ref{fig:mod34} shows $Ond_n^k$ for $k=1,2$ and $n=1,2$.  We
%see that $Ond^1_3$ is a $3\times3$ square and has iteration count 3,
%and $Ond^2_3$ is a $6\times6$ square with 12 iterations.  Similarly,
%$Ond^1_4$ is $2\times2$ with a single iteration, and $Ond^2_4$ is
%$4\times4$ with 4 iterations. The first element, 0, of each spiral in the
%figures is in a box and last element of the spiral marked with a dot.

The author generated several spirals with pencil and paper in a
methodical manner to create sets of complete spirals for various
values of $n$ and $k$.  While tedious and not exactly related to the
initial goal of looking at diagonal patterns, it helped to realize
there seemed to be patterns found in the construction of the spirals.
Specifically in the sizes of the complete spirals, iteration counts,
and where the last lattice point rested; this led the author to
investigate what the patterns were.

\subsection{On the side lengths and iterate counts of $Ond^k_n$}

In order to investigate these patterns, the author determined that
more data was needed and, due to the tedium of pencil and paper, a
program should be written to generate complete spirals
\cite{PySquare}.  This allowed the author to generate a larger number
of complete spirals and collect data on lengths, iterations, and
ending points.

In looking at the initial data, for small choices of $n, k$, a few
patterns in tuples of (lengths, iterations) were found, including
$(kn, k^2 n)$, $(\frac{kn}{2}, \frac{k^2 n}{4})$, and $(\sqrt{n}k,
k^2)$.  To establish the pattern, the prime factorization of $n$ for
each complete spiral $Ond^k_n$ with the lengths and iteration data was
generated for analysis.  From this data, the author determined there
was a relation involved with finding the greatest square divisor of
$n$, which led to the following observations.

\begin{thm}%[Length of Sides of $Ond^k_n$]
\label{lenthm}
Let $s$ denote the greatest square divisor of $n$.
The complete spiral $Ond^k_n$ has the following structure:
\begin{enumerate}[(i)]
\item If $\lambda$ is the length of the sides of $Ond^k_n$, then
\begin{equation}
  \lambda = \frac{kn}{\sqrt{s}}.
\label{lambda}
\end{equation}
\item If $\xi$ is the iteration count of $Ond^k_n$, then
\begin{equation}
  \xi = \frac{k^2n}{s}.  
\label{xi}
\end{equation}
\item If $l_{max} \in L$ is the last lattice point in the complete
  spiral $Ond^k_n$, then $l_{max}$ is either the top-right corner or
  bottom-left corner of the square.  If both $n$ and $k$ are odd, then
  $l_{max}$ will be the top-right corner of $Ond^k_n$.  In all other
  cases, $l_{max}$ will be the bottom-left corner.
\end{enumerate}
\end{thm}

One will note that in the case where $n$ is square-free, then $s=1$
and \eqref{lambda} and \eqref{xi} reduce to $\lambda = k n$ and $\xi =
k^2 n$, respectively. These can be considered an upper bound for all
cases.  One might also consider these to be the least robust in the
case of length and iteration counts.

A most interesting aspect to this process of complete spiral
construction is the connection to the greatest square divisor of some
integer $n$.  To see this more clearly, choose some $n$ and construct,
by hand, the spiral $Ond^1_n$.  At this point, you know $\lambda$ and
$\xi$, so pick one, substitute $k = 1$ and solve for $s$.  So from a
constructivist approach we find the greatest square divisor.


In the process of investigating sizes and iteration counts of
$Ond^k_n$, the first author implemented a method to map the generated
spirals to grayscale images. This works best for positive integers,
$n$, less than 256.  The map $f : \ZZ_n \to \ZZ_{256}$ is defined by a
scaled floor function,
\[
   f(j) = \alpha j, \quad \text{where} \quad
   \alpha = \left\lfloor \frac{255}n \right\rfloor
\]
The function $f$ thus maps $\ZZ_n$ to a set of brightness values which
are used to generate grayscale images.


An interesting visual example is when one generates  $Ond^{39}_{31}$. One can
see a complex pattern emerging.  Further, if one zooms
in and out on the image, various different patterns emerge that seem
to play tricks on the eye.  The image may also be view at
\cite{GraySquare} and is the file \textit{N=31\_k=39-grey.png} which
makes it easier for testing the zoom in/out.  Further, at this
location, one may find the generated images for $Ond^k_n$ for $n=2, 3,
\ldots, 31$ and $k=1,2, \ldots, 50$.

\begin{proof}[Proof of the Theorem]
  Recall that $n\ge 2$ is fixed.  Let $\lambda$ be the length of the
  side of a complete spiral, and let $\xi$ be the corresponding
  iteration count.  Since the spiral is square, we must have $\xi n =
  \lambda^2$, and so $n \vert \lambda^2$.  Now any prime which divides
  $n$ must also divide $\lambda^2$, and must thus divide $\lambda$.

  Since $s$ is the greatest square divisor of $n$, we can uniquely
  write $n = q^2_1 q_2$ where $q_2$ is square-free, with $s = q_1^2$.
  It follows that $q_1^2\vert\lambda^2$, so $q_1|\lambda$ and $q_2 \vert
  (\frac{\lambda}{q_1})^2$.  Since $q_2$ is square-free, $q_2\vert (\frac{\lambda}{q_1})$
  and we have $\lambda = k q_1 q_2$ for some integer $k$.  Clearly
  $n\vert\lambda^2$ for any such $\lambda$, and we conclude that $\lambda
  = k q_1 q_2$ is the side of the $k$-th complete spiral $Ond^k_n$.

  Equation \eqref{lambda} now follows since $\sqrt{s} = q_1$, and
  \eqref{xi} follows since $\xi = \frac{\lambda^2}{n}$.  Finally, $(iii)$
  follows since $l_{max} = \lambda^2$ is even unless $k$, $q_1$ and
  $q_2$ are all odd, and by construction $l_{max}$ represents the
  top-right corner of the square if it is odd, and the bottom-left
  corner if it is even.
\end{proof}


\section{Generalizations}

The second author proposed investigating the idea how of $Ond$ might
work for cubes or higher dimensional squares, since the lattice is the
basis for the spiral. In imagining dimension $d = 3$ and $d=4$, finding
the route in the lattice to properly spiral can be quite difficult, let alone an algorithm 
for any $d$. 

In terms of planar spirals described above, we are interested
in finding squares, thus the areal requirement $n \vert \lambda^2$. For the case
of cubic spirals, we have the volume requirement $n \vert \lambda^3$ and $n 
\vert \lambda^4$ for the $4d$ hypercube. Thus we think about the $Ond_n$ for any 
dimension $d$ as having this same geometric requirement.

\begin{thm}
\label{dimdthm}
Let $n\vert \lambda^d$ for $d=2, 3, \ldots$. Let $n = qm^d$ where $m \in \mathbb{N}$ and $q$ is $d$-power 
free. Let $q$ be defined in terms of it's prime factors, $p_j$, as
\[
	q = \prod\limits_{j=1}^{r} p_j^{e_j}\quad \text{with}\ \text{each}\quad e_j < d.
\]
then
\[
	\lambda = k m \prod p_j \quad k = 1, 2, 3, \ldots
\]
\end{thm}

\begin{proof}
Since we have that $n \vert \lambda^d$, then it follows that
\begin{align*}
  & qm^d \vert \lambda^d\\
\implies & m^d \vert \lambda^d\\
\implies & m \vert \lambda 
\end{align*}
which implies that writing $\frac{\lambda}{m}$ makes sense. Thus we have 
that $q \vert (\frac{\lambda}{m})^d$. Recall the definition of $q$ and it's clear
that $p_j \vert q$. Then because $q \vert (\frac{\lambda}{m})^d$, it follows that
$p_j \vert (\frac{\lambda}{m})^d$.  Since $p_j$ is clearly $d$-power free, $p_j \vert \frac{\lambda}{m}$.
Since the $p_j$'s are distinct, it follows that $p_1 \dots p_r \vert \frac{\lambda}{m}$ 
and thus $\lambda = k m \prod p_j$ for some $k = 1, 2, \ldots$.
\end{proof}

\section{Other Planar Shapes}

We can expand away from investigating the square spirals to other planar shapes
defined on a square lattice. While the spiraling process may vary, the area
requirement of achieving $Ond$ remains the same.

\subsection{Triangle Numbers}

In order to investigate triangle $Ond$, we make use of the triangle numbers \cite{TriangleNumbers}
to determine the area requirement. It is well known that the triangle numbers are of the form:
\[
	Area(\triangle d) =  T_d =  \frac{d(d+1)}{2}
\]
and grow as shown in Figure \ref{fig:trianglegrowth}. Thus we look at the requirement 
$n \vert \frac{d(d+1)}{2}$. Due to the factors $d$ and $d+1$, it is reasonable to suspect
that if $n \vert T_d$ then either $n \vert T_{d-1}$ or $n \vert T_{d+1}$. However, it is not
always the case that $n$ will divide consecutive triangles.

\begin{figure}[h]
\[  \begin{array}{cc}
\bullet & \ \\
\bullet & \bullet
\end{array} 
\rightarrow
%
\begin{array}{ccc}
\bullet & \  & \ \\
\bullet & \bullet \\
\bullet & \bullet & \bullet 
\end{array}
\rightarrow
\begin{array}{cccc}
\bullet & \  & \  & \  \\
\bullet & \bullet & \  & \  \\
\bullet & \bullet & \bullet & \  \\
\bullet & \bullet & \bullet & \bullet
\end{array}
\rightarrow
\begin{array}{ccccc}
\bullet & \  & \  & \  & \  \\
\bullet & \bullet & \   & \  & \  \\
\bullet & \bullet & \bullet  & \ & \  \\
\bullet & \bullet & \bullet & \bullet & \  \\
\bullet & \bullet & \bullet & \bullet & \bullet
\end{array}
\]
\caption{Triangles on Lattice $d = 2, 3, 4, 5$}
\label{fig:trianglegrowth}
\end{figure}

To more clearly see this, we generate the values 
\[
	\{ T_d \}_{d=2}^{21} = \{ 3, 6, 10, 15, 21, 28,   36, 45, 55, 66, 78,   91, 105, 120, 136, 153, 171,190,210,231 \}
\]
and observe when $n \vert T_d$, for some fixed $n \ge 2$. Here we briefly look at It is clear that only the even valued $T_d$ are $Ond_2$
achievable, which means that $T_3, T_4, T_7, T_8, T_{11}, T_{12}, T_{15}, T_{16},  T_{19}, T_{20}$ are such 
examples. From this limited sample, one sees they do appear to obey the consecutive triangle idea. From a  similar
analysis, one will see that this consecutive triangle notion holds for $n = 3, 4, 5, 7, 8, 9, 11, 12, 13, 16$.

However, the spiral $Ond_6$ is achieved by $T_3, T_8, T_{11}, T_{12}, T_{15}, T_{20}, T_{23}, T_{24}, \ldots$. In this
case,  one does see the  consecutive triangles, such as in $d = 11, 12$. However, one also sees also other cases in
which there are lone triangles. It seems to be that given a consecutive pair that achieve $Ond_6$, $T_j$ and $T_{j+1}$, 
then so will the triangles $T_{j+4}, T_{j+9}, T_{j+12}, T_{j+13}, \ldots$.

The case of $Ond_{10}$ is another case in which there are lone triangles among the various consecutive triangles
that achieve this spiral. It seems in this case, given a consecutive pair that achieves the spiral, $T_{j}, T_{j+1}$, then
so will the following triangles $T_{j+5}, T_{j+16}, T_{j+20}, T_{j+21}, \ldots$. The authors  calculated that the cases
$n = 14, 15$ also contain lone triangles among consecutive, however, have not characterized the pattern in general.

There are a few ways to visualize these triangles. The way we are interested in is in spirals, which can be seen \cite{trivisspiral}. 
On the other hand one could start in a corner and  lap back-and-forth in either 1 or 2 directions; these visualizations may 
be found \cite{trivis1} and \cite{trivis2}.



\subsection{Hexagonal Numbers}

In a similar manner to the hexagons case, we can make use of hexagonals on a lattice and 
the hexagonal numbers  \cite{HexagonalNumbers}. In a similar manner to the triangles, we can look at the
generated hexagonal numbers to determine when we can achieve \emph{Ond} with a given hexagon. The hexagonal numbers are
defined as 
\[
	H_d = d(2d - 1)
\]
Thus we are interested in the requirement $n \vert H_d$. It is necessarily the case that $2d - 1$ is 
odd valued for all $d$ of concern. Thus, if $d$ is even, then $H_d$ is even, and if $d$ is odd, then
$H_d$ is odd. It is also the case that if $n \vert d$, then $Ond_n$ can be achieved by $H_d$. Thus,
if $n \vert d$, then $H_{kd} = kd(2kd - 1)$ for $k = 1, 2, \ldots$ are all $Ond_n$ achievable. Note that 
this is a subset of possible hexagons that achieve $Ond_n$. Clearly, it is also the case that if $n \vert
2d - 1$, then $H_d$ achieves $Ond_n$. 

From numerical calculations, one sees some patterns in the $d$ values for $Ond_n$ achievable
hexagons. For example, in looking at $Ond_{11}$, the hexagons $H_{6}, H_{11}, H_{17}, H_{22}, H_{28}, \ldots$.
The pattern is $d = 6 \xrightarrow{+5} 11 \xrightarrow{+6} 17 \xrightarrow{+5} 22 \xrightarrow{+6} \ldots$.
Similar alternating patterns are found for other $Ond_n$ including $Ond_{10}$ which has 
$d = 8 \xrightarrow{+2} 10 \xrightarrow{+8} 18 \xrightarrow{+2} 20 \xrightarrow{+8}  28 \xrightarrow{+2} \ldots$ for the hexagons
$H_{8}, H_{10}, H_{18}, H_{20}, H_{28}, \ldots$. It is interesting to note that the sum of consecutive differences
is the value of $n$ in both the $n = 10, 11$ cases.

However, not all cases have this feature as in $n = 15, 16$. For $n = 15$, the pattern is over more
than two values  as in $H_{18}, H_{20}, H_{23}, H_{30}, H_{33}$ which follows 
$d = 18 \xrightarrow{+2} 20 \xrightarrow{+3} 23 \xrightarrow{+7} 30 \xrightarrow{+3} 33 \xrightarrow{+2} \ldots$. The pattern repeats and the sum
over the $d$ differences is 15, which is the value of $n$, obviously ignoring the last +2. In the case of $n = 16$, difference between $d$ values is constant 16.

The one concern is that the hexagonal numbers do not nicely fill in the lattice. That is, there are gaps in construction when
visualizing these numbers as spirals. One can see examples of these visualizations at \cite{hexvis}.

There are similar shape numbers that are defined on the planar lattice that can be explored. These include the 
hex numbers \cite{HexNum}, heptagonal numbers \cite{Hepta}, and other polygonal numbers \cite{Polyg}.

\subsection{Centered Hexagonal Numbers}

Similar to the triangles above, we can make use of the centered hexagonal numbers as described by
\cite{oeisCHex}.  These numbers can be described by
\[
	CH_d = 3d(d+1)+1 = 3d^2 + 3d + 1
\]

The first numbers are
\[
  	CH_i = \{ 7, 19, 37, 61, 91, \ldots \}
\]

One notices right away that these values are not divisible by many values and, thus, only few values of $n$ will generate
\emph{Ond}. With $n = 2, \ldots, 6$ we will not produce any \emph{Ond}, but with $7$ we will. When generating spirals for
centered hexagons, in practice, we were able to achieve \emph{Ond} for $n = 7, 13, 19, 31, 37, 43, \ldots$. 

A nice property of these hexagons are that they fill the lattice in completely within the borders of the centered hexagon.
This makes it nice for visualization and follow along with the triangle numbers and square \emph{Ond}, unlike those in 
the previous section. While there are some different patterns to be seen in these images, they are quite similar in nature
to the original square lattice images.

Visualization of these may be found at \cite{chexvis}.
\section{Summary}
This note presented the process of constructing (mod n) spirals in a square lattice and proof regarding some 
notions of patterns found in this construction. Further, we have presented a generalization of the notion of
achieving a spiral $Ond$ to hypercubes of dimension $d$ by conforming to a volume geometric requirement.
Lastly, we began to explore other $d = 2$ shapes in the $\mathbb{Z}^2$ lattice, pushing the notion of the 
geometric requirement to other forms. All code may be found at \cite{PySquare}.

\vspace{12pt}\noindent\textbf{Acknowledgments:}\quad
The first author recognizes Veracode Inc for providing support and Jared Carlson of Veracode Research for 
fruitful discussion.

\begin{thebibliography}{1}
\bibitem{Ulam} Wolfram Mathworld, ``Prime Spirals'',
  \url{http://mathworld.wolfram.com/PrimeSpiral.html}
\bibitem{PySquare} A. Reiter,
  \url{https://github.com/cwcomplex/modNspirals}
\bibitem{GraySquare} A. Reiter,
  \url{https://github.com/cwcomplex/modNspirals/tree/master/visual/square}
\bibitem{TriangleNumbers} Wolfram Mathworld, ``Triangle Numbers'',
  \url{http://mathworld.wolfram.com/TriangularNumber.html}
\bibitem{trivis1} A. Reiter, Triangles Visual One-way, \url{https://github.com/cwcomplex/modNspirals/tree/master/visual/triangles-oneway}
\bibitem{trivis2} A. Reiter, Triangles Visual Two-way, \url{https://github.com/cwcomplex/modNspirals/tree/master/visual/triangles-twoway}
\bibitem{trivisspiral} A. Reiter Triangles Visual Spiral, \url{https://github.com/cwcomplex/modNspirals/tree/master/visual/triangles-spiral}
\bibitem{oeisCHex} On-line Encyclopedia of Integer Sequences, ``Centered Hexagonal Numbers'',
 \url{https://oeis.org/A003215}
\bibitem{chexvis} A. Reiter, Centered Hexagonal Visual, \url{https://github.com/cwcomplex/modNspirals/tree/master/visual/hexagons-spiral}
\bibitem{HexagonalNumbers} Wolfram Mathworld, ``Hexagonal Numbers'',
  \url{http://mathworld.wolfram.com/HexagonalNumber.html}
\bibitem{HexNum} Wolfram Mathworld, ``Hex Numbers'',
  \url{http://mathworld.wolfram.com/HexNumber.html}
\bibitem{hexvis} A. Reiter, Hexagonal Number Visual, \url{https://github.com/cwcomplex/modNspirals/tree/master/visual/hexagons-oneway}
\bibitem{Hepta} Wolfram Mathworld, ``Heptagonal Numbers'',
  \url{http://mathworld.wolfram.com/HeptagonalNumber.html}
\bibitem{Polyg} Wolfram Mathworld, ``Polygonal Numbers'',
  \url{http://mathworld.wolfram.com/PolygonalNumber.html}
\end{thebibliography}




\end{document}  
